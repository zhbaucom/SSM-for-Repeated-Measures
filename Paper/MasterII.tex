%%%%%%%%%%%%%%%%%%%%%%%%%%%%%%%%%%%%%%%%%%%%%%%%%%
%  JASA LaTeX Template File
%  To make articles using JASA.cls, Version 1.1
%  September 14, 2019
%%%%%%%%%%%%%%%%%%%%%%%%%%%%%%%%%%%%%%%%%%%%%%%%%%

%% Step 1:
%% Uncomment the style that you want to use:

%%%%%%% For Preprint
%% For manuscript, 12pt, one column style

\documentclass[preprint]{JASA}

%%%%% Preprint Options %%%%%
%% The track changes option allows you to mark changes
%% and will produce a list of changes, their line number
%% and page number at the end of the article.
%\documentclass[preprint,trackchanges]{JASA}


%% NumberedRefs is used for numbered bibliography and citations.
%% Default is Author-Year style.
%% \documentclass[preprint,NumberedRefs]{JASA}

%%%%%%% For Reprint
%% For appearance of finished article; 2 columns, 10 pt fonts

% \documentclass[reprint]{JASA}

%%%%% Reprint Options %%%%%

%% For testing to see if author has exceeded page length request, use 12pt option
%\documentclass[reprint,12pt]{JASA}


%% NumberedRefs is used for numbered bibliography and citations.
%% Default is Author-Year style.
% \documentclass[reprint,NumberedRefs]{JASA}

%% TurnOnLineNumbers
%% Make lines be numbered in reprint style:
% \documentclass[reprint,TurnOnLineNumbers]{JASA}

\usepackage{natbib}


% Pandoc syntax highlighting

% Pandoc citation processing



\begin{document}
%% the square bracket argument will send term to running head in
%% preprint, or running foot in reprint style.

\title[A subtitle goes on another line]{This is a title and this is too}

% ie
%\title[JASA/Sample JASA Article]{Sample JASA Article}

%% repeat as needed

\author{Author's name}
% ie
%\affiliation{Department1,  University1, City, State ZipCode, Country}
\affiliation{Author's affiliation}
%% for corresponding author
\email{e-mail@uni.edu}
%% for additional information
\thanks{other info}
\author{Second author's name}
% ie
%\affiliation{Department1,  University1, City, State ZipCode, Country}
\affiliation{Second author's affiliation}
%% for corresponding author

%% for additional information


% ie
% \author{Author Four}
% \email{author.four@university.edu}
% \thanks{Also at Another University, City, State ZipCode, Country.}

%% For preprint only,
%  optional, if you want want this message to appear in upper left corner of title page
\preprint{Author-name, JASA}

%ie
%\preprint{Author, JASA}

% optional, if desired:
%\date{\today}
\date{\today}

\begin{abstract}
% Put your abstract here. Abstracts are limited to 200 words for
% regular articles and 100 words for Letters to the Editor. Please no
% personal pronouns, also please do not use the words ``new'' and/or
% ``novel'' in the abstract. An article usually includes an abstract, a
% concise summary of the work covered at length in the main body of the
% article.
Put your abstract here. Abstracts are limited to 200 words for regular
articles and 100 words for Letters to the Editor. Please no personal
pronouns, also please do not use the words
\texttt{new\textquotesingle{}\textquotesingle{}\ and/or}novel'\,' in the
abstract. An article usually includes an abstract, a concise summary of
the work covered at length in the main body of the article.
\end{abstract}

%% pacs numbers not used

\maketitle

%  End of title page for Preprint option --------------------------------- %

%% See preprint.tex/.pdf or reprint.tex/.pdf for many examples


%  Body of the article


% -------------------------------------------------------------------------------------------------------------------
%   Appendix  (optional)

%\appendix
%\section{Appendix title}

%If only one appendix, please use
%\appendix*
%\section{Appendix title}


%=======================================================

%Use \bibliography{<name of your .bib file>}+
%to make your bibliography with BibTeX.

%=======================================================

\renewcommand\refname{Model Validation}
\bibliography{bibliography.bib}


\end{document}
